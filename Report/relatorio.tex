%%%%%%%%%%%%%%%%%%%%%%%%%%%%%%%%%%%%%%%%%%%%%%%%%%%%%%%%%%%%%%%%%%%%%%%%%%%%%%%%%%%%%%
% Modelo de relatório de Disciplina de MLP a partir da
% classe latex iiufrgs disponivel em http://github.com/schnorr/iiufrgs
%%%%%%%%%%%%%%%%%%%%%%%%%%%%%%%%%%%%%%%%%%%%%%%%%%%%%%%%%%%%%%%%%%%%%%%%%%%%%%%%%%%%%%

%%%%%%%%%%%%%%%%%%%%%%%%%%%%%%%%%%%%%%%%%%%%%%%%%%%%%%%%%%%%%%%%%%%%%%%%%%%%%%%%%%%%%%
% Definição do tipo / classe de documento e estilo usado
%%%%%%%%%%%%%%%%%%%%%%%%%%%%%%%%%%%%%%%%%%%%%%%%%%%%%%%%%%%%%%%%%%%%%%%%%%%%%%%%%%%%%%
%
\documentclass[rel_mlp]{iiufrgs}

%%%%%%%%%%%%%%%%%%%%%%%%%%%%%%%%%%%%%%%%%%%%%%%%%%%%%%%%%%%%%%%%%%%%%%%%%%%%%%%%%%%%%%
% Importação de pacotes
%%%%%%%%%%%%%%%%%%%%%%%%%%%%%%%%%%%%%%%%%%%%%%%%%%%%%%%%%%%%%%%%%%%%%%%%%%%%%%%%%%%%%%
% (a A seguir podem ser importados os pacotes necessários para o documento, de acordo 
% com a necessidade)
%
\usepackage[brazilian]{babel}	    % para texto escrito em pt-br
\usepackage[utf8]{inputenc}         % pacote para acentuação
\usepackage{graphicx}         	    % pacote para importar figuras
\usepackage[T1]{fontenc}            % pacote para conj. de caracteres correto
\usepackage{times}                  % pacote para usar fonte Adobe Times
\usepackage{enumerate}              % para lista de itens com letras
\usepackage{breakcites}
\usepackage{titlesec}
\usepackage{enumitem}
\usepackage{titletoc}               
\usepackage{listings}			    % para listagens de código-fonte
\usepackage{mathptmx}               % p/ usar fonte Adobe Times nas formulas matematicas
\usepackage{url}                    % para formatar URLs
%\usepackage{color}				    % para imagens e outras coisas coloridas
%\usepackage{fixltx2e}              % para subscript
%\usepackage{amsmath}               % para \epsilon e matemática
%\usepackage{amsfonts}
%\usepackage{setspace}			    % para mudar espaçamento dos parágrafos
%\usepackage[table,xcdraw]{xcolor}  % para tabelas coloridas
%\usepackage{longtable}             % para tabelas compridas (mais de uma página)
%\usepackage{float}
%\usepackage{booktabs}
%\usepackage{tabularx}
%\usepackage[breaklinks]{hyperref}

%%
%% Julia definition (c) 2014 Jubobs
%%
\lstdefinelanguage{Julia}%
  {morekeywords={abstract,break,case,catch,const,continue,do,else,elseif,%
      end,export,false,for,function,immutable,import,importall,if,in,%
      macro,module,otherwise,quote,return,switch,true,try,type,typealias,%
      using,while},%
   sensitive=true,%
   alsoother={\$},%
   morecomment=[l]\#,%
   morecomment=[n]{\#=}{=\#},%
   morestring=[s]{"}{"},%
   morestring=[m]{'}{'},%
}[keywords,comments,strings]%

\lstset{%
    language         = Julia,
    basicstyle       = \ttfamily,
    % keywordstyle     = \bfseries\color{blue},
    % stringstyle      = \color{magenta},
    % commentstyle     = \color{ForestGreen},
    showstringspaces = false,
}

\usepackage[alf,abnt-emphasize=bf]{abntex2cite}	% pacote para usar citações abnt

%%%%%%%%%%%%%%%%%%%%%%%%%%%%%%%%%%%%%%%%%%%%%%%%%%%%%%%%%%%%%%%%%%%%%%%%%%%%%%%%%%%%%%
% Macros, ajustes e definições
%%%%%%%%%%%%%%%%%%%%%%%%%%%%%%%%%%%%%%%%%%%%%%%%%%%%%%%%%%%%%%%%%%%%%%%%%%%%%%%%%%%%%%
%

% define estilo de parágrafo para citação longa direta:
\newenvironment{citacao}{
    %\singlespacing
    %\footnotesize
    \small
    \begin{list}{}{
        \setlength{\leftmargin}{4.0cm}
        \setstretch{1}
        \setlength{\topsep}{1.2cm}
        \setlength{\listparindent}{\parindent}
    }
    \item[]}{\end{list}
}

% adiciona a fonte em figuras e tabelas
\newcommand{\fonte}[1]{\\Fonte: {#1}}

% Ative o seguinte caso alguma nota de rodapé fique muito longa e quebre entre múltiplas
% páginas
%\interfootnotelinepenalty=10000

%%%%%%%%%%%%%%%%%%%%%%%%%%%%%%%%%%%%%%%%%%%%%%%%%%%%%%%%%%%%%%%%%%%%%%%%%%%%%%%%%%%%%%
% Informações gerais                                   
%%%%%%%%%%%%%%%%%%%%%%%%%%%%%%%%%%%%%%%%%%%%%%%%%%%%%%%%%%%%%%%%%%%%%%%%%%%%%%%%%%%%%%

% título
\title{Grupo Furogga \\ Projeto Frogger Utilizando Julia \\ } 

% autor
\author{Timm do Espirito Santo}{Augusto} % {sobrenome}{nome}
\author{Stein Brito}{Eduardo} % {sobrenome}{nome}
\author{da Silva FLores}{Lucas} % {sobrenome}{nome}
%\author{Autor2}{Aluno} % {sobrenome}{nome} 1 para cada aluno

% Professor orientador da disciplina
\advisor[Prof.~Dr.]{Mello Schnorr}{Lucas}

% Nome do(s) curso(s):
\course{Curso de Graduação em Ciência da Computa{\c{c}}{\~a}o e Engenharia de Computação}

% local da realização do trabalho 
\location{Porto Alegre}{RS} 

% data da entrega do trabalho (mês e ano)
\date{04}{2018}


% Palavras chave
\keyword{Palavra-chave1}
\keyword{Palavra-chave2}
\keyword{Palavra-chave3}


%%%%%%%%%%%%%%%%%%%%%%%%%%%%%%%%%%%%%%%%%%%%%%%%%%%%%%%%%%%%%%%%%%%%%%%%%%%%%%%%%%%%%%
% Início do documento e elementos pré-textuais
%%%%%%%%%%%%%%%%%%%%%%%%%%%%%%%%%%%%%%%%%%%%%%%%%%%%%%%%%%%%%%%%%%%%%%%%%%%%%%%%%%%%%%

% Declara início do documento
\begin{document}

% inclui folha de rosto 
\maketitle      

\selectlanguage{brazilian}

% Sumario
\tableofcontents



%%%%%%%%%%%%%%%%%%%%%%%%%%%%%%%%%%%%%%%%%%%%%%%%%%%%%%%%%%%%%%%%%%%%%%%%%%%%%%%%%%%%%
% Aqui comeca o texto propriamente dito
%%%%%%%%%%%%%%%%%%%%%%%%%%%%%%%%%%%%%%%%%%%%%%%%%%%%%%%%%%%%%%%%%%%%%%%%%%%%%%%%%%%%%

%espaçamento entre parágrafos
%\setlength{\parskip}{6 pt}

\selectlanguage{brazilian}



%%%%%%%%%%%%%%%%%%%%%%%%%%%%%%%%%%%%%%%%%%%%%%%%%%%%%%%%%%%%%%%%%%%%%%%%%%%%%%%%%%%%%
% Introdução
%
\chapter{Introdução} \label{intro}



O objetivo deste trabalho consiste em estudar uma linguagem de programação moderna com características híbridas permitindo o aprendizado dos princípios de programação relacionados com os diferentes paradigmas estudados ao longo do semestre. Além disso, proporcionar a oportunidade de testar a capacidade de analisar e avaliar linguagens de programação, seguindo os critérios abordados em aula.

A tarefa principal do trabalho consiste em experimentar e comparar as características e funcionalidades orientadas a objeto e funcionais da linguagem de programação escolhida. De posse de uma linguagem, é necessário escolher um problema a ser solucionado com ela. O problema será, então, implementado duas vezes na mesma linguagem: uma delas usando somente Orientação a Objetos e a outra usando somente características funcionais.

A implementação funcional do trabalho e o relatório se encontram em \url{https://github.com/pacluke/frogger_furogga}.


% \section{Linguagem Escolhida}

% A linguagem escolhida pelo grupo foi \textbf{Julia}. Esta é uma linguagem de programação de alto nível e alto desempenho para computação numérica \cite{TJL}. Ele fornece um compilador sofisticado, execução paralela distribuída, precisão numérica e uma extensa biblioteca de funções matemáticas.

\section{Problema a Ser Resolvido}

Este trabalho tem como problema a ser resolvido o desenvolvimento do jogo Frogger. Este é um jogo \textit{arcade} que consiste em um personagem - um sapo - que deve atravessar uma estrada e um rio. O objetivo do jogo é atravessar estes lugares, desviando de obstáculos e por fim ocupar os espaços presentes no parte superior da tela, como visto na figura \ref{fig:figura1}.

 \begin{figure}[htb]
     \centering
     \caption{Tela do jogo Frogger}
     \fbox{
         \includegraphics[width=5cm, keepaspectratio]{images/frogger-screenshot.png}
     }
     \label{fig:figura1}
     \fonte{http://www.classicgaming.cc/classics/frogger/about}
 \end{figure}
 
 \chapter{Visão Geral da Linguagem}
 
 A linguagem escolhida pelo grupo foi \textbf{Julia}. Esta é uma linguagem de programação de alto nível e alto desempenho para computação numérica \cite{TJL}. Ele fornece um compilador sofisticado, execução paralela distribuída, precisão numérica e uma extensa biblioteca de funções matemáticas.
 
 \section{Paradigma Funcional}
 
 Ao trabalhar com Julia focando no paradigma funcional, percebe-se que a linguagem possui bastante suporte para esse paradigma, mas não foi construída com o propósito de ser uma linguagem funcional. Ao mesmo tempo que a linguagem facilita para o usuário a criação de funções puras sem efeitos colaterais ela também dificulta na criação de estruturas de dados imutáveis e no suporte para cópia dessas estruturas, o que vai contra uma das principais características das linguagens funcionais.
 
\section{Visão Orientada à Objetos}
 
 Julia não possui um bom suporte a orientação a objetos. Na verdade, não existe nesta linguagem nem o conceito de classes de forma explícita, como possuem linguagens orientadas a objetos como Java . No entanto, existe a possibilidade nesta linguagem de tipos abstratos. Além disso, nesta linguagem existem "supertipos", de forma que por exemplo, o tipo \textit{Real} tem como seu "supertipo" \textit{Number}. É importante ressaltar que esses tipos abstratos não podem ser instanciados e servindo apenas como forma de descrever conjuntos de tipos concretos relacionados. 
 
 Por fim, a linguagem da suporte a Tipos Compostos, que é semelhante a uma \textit{struct} da linguagem C por exemplo, sendo uma 
coleção de campos nomeados, podendo ser instanciada de forma que pode ser tratada como um único valor. Todavia, em Julia, todos os valores são objetos, mas as funções não são empacotadas com os objetos nos quais elas operam, o que faz não ser possível existirem de fato "métodos" (da forma de orientação a objetos)   de um tipo de dado.

Por todos estes motivos, foram necessários juntar estes conceitos e "criar" de alguma forma a representação de uma classe com métodos, a partir destas limitações e também do uso de algumas convenções para que a orientação a objetos fosse aplicada com esta linguagem.

Uma observação a ser feita é que na linguagem Julia existe \textit{métodos} mas estes não tem relação com métodos da orientação a objetos, estes apenas são quando se define  um comportamento para uma função, e isto não está associada a uma classe ou a um objeto.

 
%  \section{Compilador \textit{Just-in-Time} (JIT)}


 \chapter{Recursos Funcionais}
 
 Os recursos utilizados para a versão funcional do projeto da disciplina foram aplicados com base na documentação da linguagem \cite{Doc} e nos estudos em \cite{FuncPart}. Todas as funções referenciadas nessa seção estão disponíveis no arquivo \textit{frogger.jl} em \url{https://github.com/pacluke/frogger_furogga}.
 
 \section{Elementos Imutáveis e Funções Puras}
 
 Funções puras são funções que não causam efeitos colaterais. Em linguagens puramente funcionais, todas as funções são puras. Em Julia, funções que não são puras também são permitidas, mas quando uma função pura é criada, podemos explicitar utilizando essa macro antes da definição:
    \begin{lstlisting}[frame=single]
    Base.@pure
    \end{lstlisting}
    
 Um exemplo de função pura explicitada utilizada no projeto é a função \textit{change$\_$tile} na linha 002:
    \begin{lstlisting}[frame=single]
    Base.@pure function change_tile(y)
    	if(y[1] > 2 && y[1] < 11)
    		return "~"
    	elseif(y[1] > 11 && y[1] < 20)
    		return ":"
    	else
    		return "."
    	end
    end
    \end{lstlisting}
    
   \textit{change$\_$tile} foi criada para manter o cenário do jogo, representado por uma matriz, toda vez que o jogador se movimenta Ou seja, dependendo do "terreno" que o jogador está, a posição que ele estava é substituída por um caractere diferente.
 
 \section{Funções Anônimas}
    
    Funções anônimas são funções que não estão vinculadas a um identificador. Além disso, funções anônimas geralmente são Funções de Ordem Superior, como é também no caso da linguagem Julia. Seguindo esta ideia, o projeto possui o uso de funções anônimas em dois momentos diferentes.
    
    Na função \textit{move$\_$frog}:
    \begin{lstlisting}[frame=single]
    position = find(x -> x == "W", m)
    \end{lstlisting}
    
    Nesse caso, uma função anonima é passada como parametro para a função \textit{find} que devolve a posição do elemento "W" dentro da matriz \textit{m}. 
    
    O segundo caso funciona de forma parecida, mas dessa vez ela assume uma função definida pelo usuário:
    
    \begin{lstlisting}[frame=single]
    m = map(x -> replace_matrix(eval_things(x), x),
                                readdlm("map3.txt"))
    \end{lstlisting}
 
 \section{Currying}
    
    A ideia de \textit{currying} é transformar uma função vários argumentos em uma cadeia de funções onde cada uma tem apenas um argumento.
    
    A utilização desse recurso em Julia é bastante direta, como se percebe na função \textit{eval\_things}:
    \begin{lstlisting}[frame=single]
    Base.@pure function eval_things(value01)
    	equal(value02) = (value01 == value02)
    	return equal
    end
    \end{lstlisting}
    
    Essa função retorna uma função que compara um valor com o valor utilizado anteriormente como parâmetro de \textit{eval\_things}. Para exemplificar o uso dessa função, podemos voltar para o exemplo:
    
    \begin{lstlisting}[frame=single]
    global m = map(x -> replace_matrix(eval_things(x), x),
                        readdlm("map3.txt"))
    \end{lstlisting}
    
    Onde para cada elemento da matriz, é retornada uma função que possui como parâmetro para comparação o elemento atual.
    
 
 \section{Pattern Matching}
 
    \textit{Pattern matching} ou casamento de padrões é a verificação por padrões dentro de uma estrutura de dados. Em linguagens funcionais como OCaml existe uma sintaxe para representar padrões e usar isso como condicional.
    
    Julia não possui por padrão uma implementação de \textit{pattern matching} ou algo parecido, e o máximo encontrado pelo grupo após pesquisas foi uma biblioteca que não é oficial e que não funcionou a versão de Julia utilizada no trabalho (0.6.2). Por esse motivo o grupo concordou em não utilizar esse recurso no desenvolvimento da versão funcional do \textit{Frogger}.
  
    % \begin{lstlisting}[frame=single]
    % \end{lstlisting}
 
 \section{Funções de Ordem Superior}
    
    Funções de ordem superior são funções que recebem funções como parâmetro e/ou retornam funções. Um exemplo utilizado no projeto foi a função \textit{replace\_matrix}:
  
    \begin{lstlisting}[frame=single]
    Base.@pure function replace_matrix(fun, ch)
    	if(fun(1)) return ":"
    	elseif(fun(2)) return "@"
    	elseif(fun(3)) return "~"
    	elseif(fun(4)) return "X"
    	else return ch
    	end
    end
    \end{lstlisting}
    
    Ela recebe como parâmetro a função \textit{fun} e um elemento \textit{ch} e aplica a função \textit{fun} aos números 1, 2, 3 e 4, caso contrário retorna \textit{ch}. Isso serve para que uma matriz tenha os valores de 1 a 4 substituídos pelos seus respectivos caracteres, ajudando a formar a matriz que controla o jogo.
    
    Outro exemplo é a função moveAllMap\textit{moveAllMap} vista abaixo.
    
    \begin{lstlisting}[frame=single]
    function moveAllMap(funGenerateOBJ,map,
    unmovableLines,j,sizej)
	if( j in unmovableLines)
		j=moveTomovable(unmovableLines,j)
	end
	if(j<sizej)
		newOBJ = funGenerateOBJ(j)
		if(j%2 == 0)
			moveLine(newOBJ,map,true
			,2,(size(m,2)-2),j)
		else
			moveLine(newOBJ,map,false
			,2,(size(m,2)-2),j)
		end
		moveAllMap(funGenerateOBJ,
		map,unmovableLines,j+1,sizej)
	else
		return
	end

    end
    \end{lstlisting}
    

    
    Essa função recebe uma função que vai gerar o novo objeto a ser inserido no mapa para a movimentação, como um carro ou um chão. Isso serve para facilitar a reutilização de código, pois assim a criação de objetos pode ser alterada em cada uso da função \textit{moveAllMap} sem maior dificuldades.
   
 
    \section{Funções de Ordem Maior Fornecidas Pela Linguagem (Map)}
 
    Julia fornece várias funções de ordem maior como \textit{reduce}, \textit{foldl}, \textit{foldr}, entre outras. Foi escolhido pelo grupo a função \textit{map}:
  
    \begin{lstlisting}[frame=single]
    global m = map(x -> replace_matrix(eval_things(x), x),
                        readdlm("map3.txt"))
    \end{lstlisting}
    
    A função \textit{map} aplica, para cada elemento da matriz \textbf{m}, uma função que substitui cada elemento da matriz (número) pelo seu caractere equivalente.
 
 \section{Funções como Elementos de Primeira Ordem}
 
    Funções são utilizadas como elementos de primeira ordem quando elas podem ser recebidas como parâmetro e também retornadas por outras funções, assim como já foi demonstrado na função \textit{replace\_matrix} que recebe uma função como parâmetro e na função \textit{eval\_things} que retorna uma função.
 
 \section{Recursão}
 
    A recursão foi utilizada em uma das funções mais importantes do projeto, que reimprime a matriz do jogo toda vez que a mesma é atualizada:
  
    \begin{lstlisting}[frame=single]
    function print_map_rec(map, j, sizej, i, sizei)
    	# print da matriz
    	if(map[j, i] == "~")
    	    print_with_color(:cyan, map[j, i])
    	elseif(map[j, i] == ":")			
    		print(map[j, i])
    	elseif(map[j, i] == "@" || map[j, i] == "X")
    		print_with_color(:red, map[j, i])
    	elseif(map[j, i] == "^")
    		print_with_color(:magenta, map[j, i],
    		bold=true)
    	elseif(map[j, i] == "W")
    		print_with_color(:green, map[j, i],
    		bold=true)
    	else
    		print_with_color(:white, map[j, i])
    	end
    
    	# condicoes da recursao
    	if (i+1 > sizei && j+1 > sizej) print("\n\r")
    	    return 0
    	elseif(i+1 > sizei)
    		print("\n\r")
    		print_map_rec(map, j+1, sizej, 1, sizei)
    	else print_map_rec(map, j, sizej, i+1, sizei)
    	end
    end
    \end{lstlisting}
    
    A função \textit{print\_map\_rec} é encontrada no arquivo \textit{frogger.jl}.
    
    
\chapter{Recursos de Orientação a Objetos}

\section{Classes}

Como não existe suporte da linguagem a classes de forma explicita, esta foi definida segundo alguns critérios e implementações. Segue abaixo um exemplo de uma classe:
    \begin{lstlisting}[frame=single]

abstract type GameObjectClassType end  

type GameObject <: GameObjectClassType
	_x_pos::Int # private denoted by the underscore '_'
	_y_pos::Int 
	_symbol::Char 
	_size::Int 
	get_symbol::Function # Public
	get_size::Function # Public
	get_x_pos::Function # Public
	get_y_pos::Function # Public
	set_x_pos::Function # Public
	set_y_pos::Function # Public
	GameObject(x, y, symbol, size) = 
	new(x, y, symbol, size, get_symbol,
	get_size, get_x_pos, get_y_pos,
	set_x_pos, set_y_pos) # public constructor
end

function get_symbol(self::GameObjectClassType)
	return self._symbol
end

function get_size(self::GameObjectClassType)
    return self._size;
end

function get_x_pos(self::GameObjectClassType)
	return self._x_pos
end

function get_y_pos(self::GameObjectClassType)
	return self._y_pos
end

function set_x_pos(self::GameObjectClassType, x::Int)
	if x < 0
		throw("Error! Position cant be negative")
	end	
	self._x_pos = x
end

function set_y_pos(self::GameObjectClassType, y::Int)
	if y < 0
		throw("Error! Position cant be negative")
	end	
	self._y_pos = y
end

\end{lstlisting}
    
Cada classe é um arquivo, como neste caso acima, o arquivo game\_object.jl é uma classe. Isto foi convencionado pelo grupo, de forma que seja possível termos classes nesta linguagem. Em nossa implementação, cada classe precisa ter um tipo abstrato associado, neste caso GameObjectClassType, o que permite o mecanismo de herança e polimorfismo, como será detalhado nas próximas seções.


Além disso, existe sempre associado um Tipo Composto concreto, que neste caso é o GameObject que possuí o tipo abstrato GameObjectClassType. Esse tipo composto é que possui todos os atributos da classe além de seus métodos.

Como não é possível existirem métodos de verdade, o que foi feito é que parte dos atributos são funções e estas funções são definidas no mesmo arquivo. Desta forma, é necessário que \textbf{um dos parâmetros destas funções seja o próprio objeto do qual será chamada a função}. Por convenção, sempre este será o primeiro parâmetro e é chamado de \textbf{self}:

\begin{lstlisting}[frame=single]
function get_x_pos(self::GameObjectClassType)
	return self._x_pos
end
\end{lstlisting}



Para então vincular essas funções do arquivo com os atributos (formando assim de fato uma classe com métodos), é utilizado o construtor, em que ele associa essas funções. É importante notar que para quem usa essa classe, este não pode definir essas funções na hora de criar o objeto, pois é apenas exposto para ele atributos que de fato devem ser \textit{setados}. Isso pode ser visto abaixo:

\begin{lstlisting}[frame=single]

GameObject(x, y, symbol, size) = new(x, y, 
symbol, size,
get_symbol, get_size,
get_x_pos, get_y_pos,
set_x_pos, set_y_pos)

\end{lstlisting}

Note que só pode serem definidos x, y, symbol e size na criação do objeto.


\section{Encapsulamento e proteção dos atributos}

Como não existe nesta linguagem alguma forma de existir atributos privados, o que foi feito é utilizar uma convenção para definir que algum atributo é privado. Assim como python, foi definido um prefixo para definir que a variável é privada. Neste caso foi utilizado o prefixo "\_". Para acessar esses atributos, foram colocados funções de \textit{getter} e \textit{setter}:


\begin{lstlisting}[frame=single]

[...]

type GameObject <: GameObjectClassType
	_x_pos::Int
	_y_pos::Int
	_symbol::Char
	_size::Int
	get_symbol::Function # Public
	get_size::Function # Public
	get_x_pos::Function # Public
	get_y_pos::Function # Public
	set_x_pos::Function # Public
	set_y_pos::Function # Public
	GameObject(x, y, symbol, size) = 
	new(x, y, symbol, size, get_symbol, get_size, 
	get_x_pos, get_y_pos, set_x_pos, set_y_pos) 
	# public constructor
end

[...]

function set_x_pos(self::GameObjectClassType, x::Int)
	if x < 0
		throw("Error! Position cant be negative")
	end	
	self._x_pos = x
end

function set_y_pos(self::GameObjectClassType, y::Int)
	if y < 0
		throw("Error! Position cant be negative")
	end	
	self._y_pos = y
end

\end{lstlisting}


Caso haja uma tentativa de colocar um valor inválido, uma exceção é disparada:
\begin{lstlisting}[frame=single]

ERROR: LoadError: "Error! Position cant be negative"
Stacktrace:
 [1] set_x_pos
[...]
\end{lstlisting}


Desta forma há a validação dos parâmetros para que estejam dentro do esperado.

\section{Construtores}

Em Julia existem construtores e estes são usados para instanciar Tipos Compostos. Desta forma, na nossa implementação, os construtores na verdade instanciam um objeto do que seria uma classe. Como já descrito anteriormente, também foi utilizado o construtor para fazer o \textit{binding} entre funções do arquivo que representa a classe com as funções que são atributos do Tipo Composto, fazendo assim existirem métodos.

O uso desses construtores em uma instanciação e sua definição pode ser visto abaixo:
\begin{lstlisting}[frame=single]

[...]
# Arquivo frogger.jl
enemy1 = Car(15, 18)

[...]

# Arquivo car.jl

include("./enemy.jl")

abstract type CarClassType  <: EnemyClassType end  

type Car <: CarClassType 
    super::Enemy # 'Superclass'
	Car(x,y) = new(Enemy(x,y,'@',2,2)) # public
end

\end{lstlisting}
\section{Destrutores}

Julia possui um coletor de lixo (Garbage Collector) de forma que não é necessária a implementação de um destrutor.

\section{Espaços de nome diferenciados}


Em Julia os módulos permitem que sejam criadas definições de alto nível sem se preocupar com conflitos de nome quando o código é usado junto algum código externo. Dentro de um módulo, é possível controlar quais nomes de outros módulos estão visíveis (via importação) e especificar quais os nomes devem ser públicos (via exportação). 

Inicialmente todas as classes as quais desenvolvemos eram na verdade módulos. Desta forma, ficava fácil trabalhar com estas classes e com seus espaços de nome diferenciados, pois para usar uma classe era necessário importa-la e apenas o que foi exportado do módulo ficaria disponível. No caso de conflitos de nomes, poderia se utilizar como prefixo o nome do módulo, seguindo de ponto, como por exemplo \textbf{ModuloA.UmaClasseExemplo} e \textbf{ModuloB.UmaClasseExemplo}.

No entanto, notamos que com o uso de módulos, não era possível o funcionamento correto da herança, pois os supertipos ficavam limitados ao escopo do módulo. Explicando um pouco melhor, se por exemplo fosse criada uma lista com objetos do tipo CarClassType e TrunkClassType, essa lista deveria poder ser considerada uma lista do tipo GameObjectClassType, já que ambos os tipos são filhos deste supertipo. No entanto, devido ao uso destes módulos, um objeto da classe Car (que pertence ao tipo CarClassType)  tinha como supertipo o \textbf{CarModule.GameObjectClassType}, ou seja, seu supertipo ficava atrelado ao módulo, fazendo com que Car e Trunk não tivessem o mesmo supertipo GameObjectClassType, pois o supertipo de Trunk era \textbf{TrunkModule.GameObjectClassType}.

Para resolver este problema, não foram mais utilizados módulos e o mecanismo de herança funciona como descrito na próxima seção. A consequência disso, é que para se ter acesso a classes e seus métodos, é necessário apenas realizar o comando \textit{include}:
\begin{lstlisting}[frame=single]
include("./game_objects/car.jl")
[...]
enemy1 = Car(15, 18)
[...]
\end{lstlisting}

No caso de conflito de nomes, o último incluído é o válido. Isso faz que o uso dos \textit{includes} seja feito com mais cautela, para evitar esse tipo de problema.

\section{Mecanismo de herança}

Para o funcionamento do mecanismo de herança, usa-se o mecanismo da linguagem de \textit{supertipos}. Para uma classe estender a outra, essa deve incluir (comando \textit{includes}) o arquivo da mesma e seu Tipo Concreto deve ser um subtipo de um tipo abstrato \textbf{deste} módulo, que é um \textbf{subtipo do tipo abstrato da classe pai}. Explicando um pouco melhor, se a classe EnemyClass herda de GameObject, esta necessariamente precisa que seu tipo abstrato interno EnemyClassType seja um subtipo de GameObjectClassType e que o tipo concreto Enemy seja do tipo EnemyClassType. Isso permite que por exemplo, alguma função que receba por parâmetro um GameObjectClassType, possa receber um objeto EnemyClassType, já que este é um subtipo de GameObjectClassType. Isso ficará mais evidente no trecho que código que mostrado posteriormente.

Para que os atributos sejam herdados da classe pai, a única forma encontrada pelo grupo de isso funcionar corretamente é que um dos atributos do filho seja um atributo que por convenção foi chamado de \textit{super}, que nada mais é do que um objeto do Tipo Concreto da classe pai. Assim, o construtor do filho deve chamar (de forma implícita para quem quer instanciar essa classe) o construtor do pai. Isso pode ser visto no trecho abaixo:

\begin{lstlisting}[frame=single]
type Enemy  <: EnemyClassType 
    super::GameObject # 'Superclass'
    _max_speed::Int
    move::Function # Public method
	Enemy(x, y, symbol, size, speed) =
	new(GameObject(x, y, 
	symbol, size), 
	speed, move) # public
end
\end{lstlisting}

Para acessar os atributos e métodos da \textit{superclasse}, necessariamente precisa-se acessar de forma semelhante a seguinte:
\begin{lstlisting}[frame=single]
objeto.super.metodo_herdado(objeto.super, param1, param2)
\end{lstlisting}


Existem também classes abstratas. Estas, por definição, não podem serem instanciadas, apenas estendidas por uma classe concreta que então pode ser instanciada. Para que isso fosse possível em Julia, uma classe abstrata, nada mais é que um arquivo que possui também um tipo abstrato associado, também com funções, mas não existe um Tipo Composto, de forma que então este não pode ser instanciado diretamente. Apenas quem herda pode instanciar e usar as funções definidas nesta classe abstrata. 

Uma grande diferença, devido a nossa implementação, é que como não há um Tipo Composto não há atributos dessa classe abstrata que são herdados pelos filhos, é possível apenas herdar as funções de forma que os filhos devem associar essas funções como um atributo seu. Além disso, o atributo \textit{super} que aponta para a classe Pai somente pode apontar para um Avô ou apontar para ninguém, uma vez que \textit{super} é sempre um atributo de um Tipo Composto. Infelizmente o grupo não encontrou alguma outra forma de representar uma classe abstrata nessa linguagem. Segue exemplo de uma classe abstrata e de uma classe filha que a implementa:
\begin{lstlisting}[frame=single]
#CLASSE ABSTRATA
#ARQUIVO movable_object.jl
include("./game_object.jl")

abstract type MovableObjectAbstractClassType 
<: GameObjectClassType  end  

MAX_SCREEN_X_SIZE = 80

function move(self::MovableObjectAbstractClassType)
    current_x_pos = self.super.get_x_pos(self.super)
    new_x_pos = current_x_pos - rand(1:2)
    if new_x_pos <= 1
        self.super.set_x_pos(self.super, MAX_SCREEN_X_SIZE 
        - self.super.get_size(self.super))
    else
        self.super.set_x_pos(self.super, new_x_pos)
    end
end


---------------------------
#CLASSE FILHA CONCRETA
#ARQUIVO floating_object.jl


include("./movable_object.jl")
include("./game_object.jl")

abstract type FloatingObjectType  
<: MovableObjectAbstractClassType end  

type FloatingObject <: FloatingObjectType 
	super::GameObject # 'Superclass'
    move::Function # Public method
	FloatingObject(x, y, symbol, size)
	= new(GameObject(x, y, symbol, size), move)
end



\end{lstlisting}

Note que FloatingObject tem o atributo \textit{move} que é o método de mover o objeto. É feito o \textit{binding} da função \textit{move} com esse atributo formando o método usando o construtor, conforme já foi descrito em seções anteriores. No entanto percebe-se que \textbf{a função \textit{move} não está definida na classe FloatingObject e sim na sua classe abstrata pai} definida no arquivo movable\_object.jl. Note que \textbf{não é possível realizar a instanciação de algum objeto da classe abstrata} do arquivo movable\_object.jl de forma a se comportar corretamente como uma classe abstrata.


O mecanismo de herança foi bastante utilizado neste trabalho, em que temos até 4 níveis de hierarquia. Temos a classe pai de todos os objetos do jogo que é a classe GameObject. O diagrama \textbf{simplificado} das classes filhas de GameObject  pode ser visto na figura \ref{fig:figura2}.

\begin{figure}[htb]
    \centering
     \caption{UML simplificado da hierarquia de GameObjects}
    \fbox{
        \includegraphics[width=10cm,height=16cm,keepaspectratio,angle=0]{images/UML.png}
    }
    \label{fig:figura2}
\end{figure}

 


\section{Polimorfismo por inclusão}

Na classe Map desenvolvida, que é responsável por mostrar o mapa colocando os objetos sobre o mesmo, é possível verificar um polimorfismo. Isso porque é chamado os métodos get\_size, get\_y\_pos, get\_x\_pos e get\_symbol de uma lista de objetos, em que estes são de classes diferentes mas todos herdam de GameObject, o qual possui esses métodos. A mesma coisa acontece com a classe controller que chama o método \textit{move} de todos os GameObjects, sem saber se estes são da classe Car, Truck ou qualquer outra, pois todas essas classes herdam GameObject.
\begin{lstlisting}[frame=single]
function move_objects(game_objects)  #private function
    for i=1:size(game_objects, 1)
        game_objects[i].super.move(game_objects[i].super)
    end
end
\end{lstlisting}
\section{Polimorfismo paramétrico}

Julia fornece suporte a Tipos Paramétricos. Estes são semelhante a Generics de Java, ou Templates em C. Basicamente os tipos podem receber parâmetros, de modo que as declarações de tipo introduzem novos tipos - um para cada combinação possível de valores de parâmetros.

No desenvolvimento deste jogo, para demonstrar o uso deste polimorfismo, mesmo que não necessário neste caso, implementamos a classe Map com o parâmetro T:
\begin{lstlisting}[frame=single]

include("./collision_manager.jl")

type Map{T} 
    _map::Array{Any,2}
    _game_objects::Array{T,1}
    get_game_objects::Function
    show::Function
    collision_manager::CollisionManager
    Map{T}(game_objects) where {T} = 
    new(readdlm("map.txt"), game_objects, 
    get_game_objects, show_map, CollisionManager())
end

[...]
\end{lstlisting}

Note que T pode ser qualquer tipo. É na chamada do contrutor de Map que este tipo é definido. Como neste caso o Array de GameObjects deve justamente ser do tipo GameObject isso é determinado na criação do objeto:
\begin{lstlisting}[frame=single]
[...]
map = Map{GameObjectClassType}(game_objects)
[...]
\end{lstlisting}

Esta \textbf{implementação não era necessária, uma vez que não foi utilizado um Map pra qualquer outro tipo de dado}. No entanto foi implementado assim apenas para \textbf{demonstrar a possibilidade do uso deste recurso polimórfico} utilizando esta linguagem.

Outro fator a se notar é que o próprio \textit{Array}, elemento básico da linguagem é um Tipo Paramétrico, uma vez que recebe dois parâmetros, o primeiro sendo o tipo do Array e o segundo suas dimensões. 

\section{Polimorfismo por sobrecarga}

Um polimorfismo de sobrecarga pode ser visto na implementação da classe Enemy. Essa herda da classe abstrata MovableObjectAbstractClass, no entanto ela implementa seu método \textit{move}, mesmo que este tenha sido herdado da classe abstrata. Em sua implementação, leva-se em conta o atributo _max_speed para mover o objeto.

\begin{lstlisting}[frame=single]
include("./movable_object.jl")
include("./game_object.jl")

abstract type EnemyClassType  
<: MovableObjectAbstractClassType end  

type Enemy  <: EnemyClassType 
    super::GameObject # 'Superclass'
    _max_speed::Int
    move::Function # Public method
	Enemy(x, y, symbol, size, speed) 
	= new(GameObject(x, y, symbol, size), speed, move) 
end

function move(self::Enemy)
    current_x_pos = self.super.get_x_pos(self.super)
    new_x_pos = current_x_pos - rand(1:max(self._max_speed))
    if new_x_pos <= 1
        self.super.set_x_pos(self.super,
        MAX_SCREEN_X_SIZE - self.super.get_size(self.super))
    else
        self.super.set_x_pos(self.super, new_x_pos)
    end
end
\end{lstlisting}

\section{Delegates}
    
Considerando como a definição do uso de \textit{delegates} poder avaliar um membro de um objeto (o receptor) no contexto de outro objeto original (o remetente), podemos verificar que isto foi implementado de uma certa forma em nosso trabalho. 

A classe Launcher por exemplo, foi modificada da forma original para exemplificar o uso de delegação. Existe o método \textit{init\_game} que apenas delega para o objeto de mapa a inicialização deste, e por consequência, a inicialização do jogo. Isso pode ser visto com mais detalhes no trecho abaixo:

\begin{lstlisting}[frame=single]

include("./controller.jl")

abstract type LauncherClassType end  

type Launcher <: LauncherClassType
    start::Function # public
    init_game::Function
    _controller::ControllerClassType
    _map::Any
    _player::Any
	Launcher() = 
	new(start, init_game, Controller(),
	nothing, nothing) # public
end

function init_game(self::LauncherClassType, map, player)
    self._map = map
    self._player = player
    self._map.show(self._map, self._player)
end

function start(self::LauncherClassType)

    while true

        run(`stty raw`)	
                       
                      
        self._controller.capture_keyboard(
        self._controller, self._player, self._map)
    end
end

\end{lstlisting}

Quem "usa" o método do Launcher não consegue ver explicitamente que foi delegado ao Map a inicialização do jogo, como pode ser visto:
\begin{lstlisting}[frame=single]
launcher.init_game(launcher, map, frog)
\end{lstlisting}
    
% %%%%%%%%%%%%%%%%%%%%%%%%%%%%%%%%%%%%%%%%%%%%%%%%%%%%%%%%%%%%%%%%%%%%%%%%%%%%%%%%%%%%%
% % Conclusões
% %
\chapter{Conclusão Final}
    
    Julia não é uma linguagem de programação funcional, no entanto a mesma fornece diversos recursos de linguagens funcionais, como por exemplo funções anônimas e funções de alta ordem. 
    Mesmo não sendo uma linguagem ideal para a programação funcional, foi possível desenvolver o jogo de forma suficientemente satisfatória, replicando todos os aspectos básicos do jogo clássico, Frogger, como a movimentação dos objetos, carros e troncos, ao longo do cenário, além da movimentação do próprio personagem.
    
    Já no desenvolvimento utilizando o paradigma de Orientação a Objetos, foi necessário bastante criatividade para conseguir construir conceitos e elementos básicos de orientação a objetos dentro da linguagem, como por exemplo, a definição de uma classe.
    
    Apesar destes contratempos, notou-se claramente que o uso de Classes e Objetos tornam mais fácil trabalhar com o domínio do problema, principalmente quando se trata de jogos. Isso porque em um jogo, é fácil definir os objetos, como por exemplo os personagens, inimigos, itens do cenário e entre outros. 
    
    Com o uso de herança, ficava ainda mais trivial o desenvolvimento de tipos diferentes de inimigos, uma vez que era possível criar uma classe pai mais genérica e filhos com suas características próprias.
    
    Realmente a grande dificuldade associada foi a falta de suporte da linguagem. No entanto, se este jogo tivesse sido desenvolvido em linguagem como Java, que é uma linguagem de fato orientada a objetos, não só teria sido mais rápido o desenvolvimento, quanto o código seria muito mais limpo, seguindo boas práticas.
    
    Isso demonstra claramente que para abordar um problema de desenvolvimento de \textit{software} na computação, não basta apenas programar, mas dedicar tempo para a escolha de uma linguagem adequada e o paradigma que melhor lida com o problema, e para tal, é necessário o conhecimento de diferente modelos de linguagens de programação.

% Apresentar conclusão do trabalho...


%\paragrafo

% Os títulos das subdivisões do texto são apresentados em fonte tamanho 12 pt, com as seguintes variações de estilo: 

% \begin{itemize}[leftmargin=3em] % [label={--}]

% \setlength{\itemindent}{1em}

%     \item \textbf{Capítulos}: fonte Helvetica, negrito, todas em maiúsculas;

%     \item \textbf{Seções}: fonte Times, negrito;

%     \item \textbf{Subseções}: fonte Times, normal. 

% \end{itemize}

% Não devem ser incluídos títulos das seções de 4o. e 5o. nível, nem o detalhamento dos Apêndices e/ou Anexos. 

% O documento atual já utiliza estilos e comandos \LaTeX\ apropriados para a construção correta do sumário. 

% No caso de o trabalho ser apresentado em mais de um volume, cada um deve conter o sumário geral da obra, bem como seu próprio sumário, ocupando páginas consecutivas. 



% \subsubsection{sobre a Lista de Abreviaturas e Siglas}

% Todas as abreviaturas e siglas devem ser ordenadas alfabeticamente e seguidas de seus respectivos significados. Um exemplo pode ser visualizado no início deste documento. 



% \subsubsection{Sobre a Lista de Símbolos}

% Semelhante à lista de abreviaturas e siglas, os símbolos utilizados no documento devem ser apresentados na ordem em que nele aparecem, acompanhados de seus respectivos significados. 



% \subsubsection{Sobre as Listas de Figuras e de Tabelas}

% Separadamente para as Figuras e Tabelas, devem ser relacionadas as ilustrações na ordem em que aparecem no texto, indicando, para cada uma, o seu número, legenda e página onde se encontra.

% O documento atual já utiliza estilos e comandos \LaTeX\ apropriados para a construção correta das listas de Figuras e Tabelas. 



% \subsection{Numeração das Páginas}

% Os números de página são colocados na margem superior do documento, a 2~cm da borda superior do papel, alinhados à {\it margem externa} do texto. Por margem externa entende-se a margem direita nas páginas ímpares e a esquerda nas páginas pares. Quando o documento é produzido somente-frente, utiliza-se sempre a margem direita para a numeração. 

% Todas as páginas do documento, a partir da folha de rosto, são contadas, mas a numeração só é mostrada a partir do primeiro capítulo de texto propriamente dito (ou seja, normalmente a Introdução). Assim, as primeiras páginas não devem apresentar numeração.

% O documento atual já utiliza estilos \LaTeX\ apropriados para a inserção correta da numeração de páginas. 



% %%%%%%%%%%%%%%%%%%%%%%%%%%%%%%%%%%%%%%%%%%%%%%%%%%%%%%%%%%%%%%%%%%%%%%%%%%%%%%%%%%%%%
% % Capítulo 2
% %
% \chapter{AS ILUSTRAÇÕES NO TEXTO}

% As ilustrações no texto são geralmente apresentadas ou como Figuras ou como Tabelas. Devem ser acompanhadas de uma legenda explicativa, na qual devem constar o tipo de ilustração (texto "Figura" ou "Tabela"), o respectivo número de ordem, e o texto que descreve a ilustração. Os números de ordem são subordinados ao capítulo onde aparecem, devendo ser apresentados na forma ``X.Y'', onde X é o número do capítulo e Y é o número de ordem da ilustração dentro do capítulo. As numerações de Figuras e Tabelas são independentes entre si. Veja exemplos de legendas nas ilustrações deste documento. 

% O documento atual já utiliza estilos \LaTeX\ apropriados para a inserção correta da formatação e numeração de Figuras e Tabelas. 


% \section{Descrição das Figuras}

% Veja exemplo de formatação da figura \ref{fig:figura1} a seguir: a legenda aparece acima da ilustração, a descrição deve ser centralizada, no número de identificação  \ref{fig:figura1}, o número 2 corresponde ao capítulo onde se localiza a figura e o número 1 a ordem da figura dentro do capítulo, seguido de dois ponto, espaço e a breve descrição da figura, que deve ter a {\bf primeira} letra em maiúsculo.


% \begin{figure}[htb]
%     \centering
%     \caption{Exemplo de apresentação de uma figura no texto}
%     \fbox{
%         \includegraphics[width=6cm,height=4cm,keepaspectratio]{images/image1.png}
%     }
%     \label{fig:figura1}
%     \fonte{xxxxxx}
% \end{figure}


% Se  buscada em alguma obra publicada, deve aparecer a fonte da figura, como no exemplo. Caso elaborada pelo próprio autor, incluir "Fonte: autor". Observando que na LISTA DE TABELAS a fonte não deve aparecer. 



% \section{Descrição das Tabelas}

% Veja exemplo de formatação da Tabela \ref{tab:tabela2}: a legenda aparece acima da tabela, a descrição deve ser centralizada, no número de identificação 2.1, o número 2 corresponde ao capítulo onde se localiza a tabela e o número 1 a ordem da tabela dentro do capítulo, seguido de dois ponto, espaço e  breve descrição, que deve ter a {\bf primeira} letra em maiúsculo. 

% Assim como figuras, tabelas também devem indicar sua fonte. Caso elaboradas pelo próprio autor, incluir "Fonte: autor". Observando que na LISTA DE TABELAS a fonte não deve aparecer. 

% Atente que {\bf as laterais das tabelas são abertas}. Isso torna a imagem mais limpa e clara. As tabelas do texto não devem exceder a margem.


% \begin{table}[ht]
%   \caption{Exemplo de apresentação de uma tabela no texto}
%   \centering
%     \begin{tabular}{ c | c | c }
%         \hline 
%         \textit{Manga} & 
%         \textit{Abacaxi} &  
%         \textit{Morango}  \\
%         \hline
%         12    & 100.000,00     & 10.000,00 \\
%         \hline
%         12    & 10.000,00     & 100.000,00 \\
%       \hline
%     \end{tabular}
%   \fonte{EREGALI, 2004. p. 356.}
%   \label{tab:tabela2}
% \end{table}



% %%%%%%%%%%%%%%%%%%%%%%%%%%%%%%%%%%%%%%%%%%%%%%%%%%%%%%%%%%%%%%%%%%%%%%%%%%%%%%%%%%%%%
% Capítulo 3
%
% \chapter{Sobre referências e citações}

% A classe \emph{iiufrgs} faz uso do pacote \emph{abnTeX2} com algumas alterações
% feitas por Sandro Rama Fiorini. Culpe ele se algo der errado. Agradeça a ele
% pelo que der certo. As modificações dão uma camada de tinta NATBIB-style,
% já que o abntex2 usa uns comandos de citação feitos para alienígenas de 5 braços (wtf!?). \\

% Exemplos de citação:
% \begin{itemize}[leftmargin=3em]

% \setlength{\itemindent}{1em}

%  \item \textbf{cite}: Unicórnios são verdes \cite{TJL};
    
%  \item \textbf{citet}: Segundo \citet{TJL}, unicórnios são verdes.
    
%  \item \textbf{citeauthoronline e citeyearpar}: Segundo artigos de \citeauthoronline{Adams2009Conceptual}, unicórnios são verdes \citeyearpar{Adams2009Conceptual}.
    
% \end{itemize}
 
 
 
% Recomenda-se o uso de bibtex para gerenciar as referências (veja o arquivo
% biblio.bib).



% \section{citações}

% Há duas formas de se fazer uma citação: a {\bf indireta} ou {\bf livre} (também chamada de {\it paráfrase}) e a {\bf citação direta} ou {\bf textual}. Pode haver, ainda, a {\bf citação de citação}.

% Todas as citações devem trazer a {\bf identificação }de sua autoria.


% \section{Citação Indireta ou Livre (paráfrase)}

% Chamamos de citação indireta ou livre ({\it paráfrase}) aquela citação na qual expressamos o {\bf pensamento de outra pessoa }com {\bf nossas próprias palavras.}

% Após fazermos a citação, devemos indicar o nome do autor, {\bf em letras minúsculas, }se estiver no corpo do texto, e com letras {\bf maiúsculas, }se estiver dentro dos parênteses, juntamente com o {\bf ano }da publicação da obra em que se encontra a idéia por nós referida. Não são indicadas páginas já que a idéia pode estar sendo resumida de uma obra inteira, de um capítulo, de diversas partes ou de um conjunto delas.

% Desta forma (com o nome no corpo do texto):

% Depois de analisar a situação, Nóvoa (1993) chegou a afirmar que o brasileiro ainda não está capacitado para escolher seus governantes por causa de sua precária vocação política e da absoluta falta de escolaridade, já que o homem do povo, o zé-povinho, geralmente não sabe sequer em quem votou nas últimas eleições, não sabe sequer quem são seus governantes, não saber sequer quem determina seu próprio meio de sobreviver.

% Ou, então, (com o nome nos parênteses):

% Depois de analisar a situação, chegou-se a afirmar que o brasileiro ainda não está capacitado para escolher seus governantes por causa de sua precária vocação política e da absoluta falta de escolaridade, já que o homem do povo, o zé-povinho, geralmente não sabe sequer em quem votou nas últimas eleições, não sabe sequer quem são seus governantes, não saber sequer quem determina seu próprio meio de sobreviver (NÓVOA, 1993).

% No caso de o autor possuir outras obras, elas serão diferenciadas pela data da publicação. Havendo mais de uma obra no mesmo ano, acrescentamos uma letra após a data.

% No caso do teatro ou do cinema quem melhor se definiu foi Antunes (1997-a) quando declarou que aqueles espaços haviam sido todos tomados pela geração de 40. Por outro lado, ele próprio se contradisse, mais tarde, (1997-b), como já se contradissera noutras ocasiões, ao referir-se às decisões tomadas pelos autores da geração de 50. Isso é uma incongruência com a qual convivemos há muito tempo.

% Quando, no transcorrer do texto, em citações indiretas ou livres, se faz menção, seguidas vezes, ao mesmo autor, na mesma obra, {\bf {\it não é necessário}}{\it  }que{\it  }se repita a indicação do ano.


% \begin{table}[ht]
%   \caption{Deve-se escolher somente um tipo de citação para usar durante o texto}
%   \centering
%     \begin{tabular}{ l | p{8cm} }
%         \hline
%         \multicolumn{2}{c}{FORMATAÇÃO DAS CITAÇÕES DOS AUTORES DURANTE O TEXTO} \\
%         \hline
%         Nóvoa (1993) & O nome do autor deve ser escrito em letras {\bf minúsculas }quando apresentado no próprio texto \\
%         \hline
%         (GUIMARÃES, 1985, p.32) & O nome do autor deve ser escrito em letras {\bf maiúsculas }quando apresentado dentro dos parênteses. \\
%       \hline
%     \end{tabular}
%   \\Fonte: MEREGALI, 2004. p. 356.
%   \label{tab:tabela3}
% \end{table}



% \section{Citação Direta ou Textual (transcrição)}

% São chamadas de citações diretas ou textuais aquelas em que se transcrevem {\bf exatamente as palavras do autor citado}. As citações diretas ou textuais podem ser {\bf breves }ou {\bf longas.}

% São consideradas {\bf breves} aquelas cuja extensão não ultrapassa {\it três linhas. }Essas citações devem {\it integrar o texto e }devem vir {\bf entre aspas. O tamanho }da {\bf fonte }(letra) da citação breve {\bf permanece }o mesmo do corpo do texto {\bf {\it (12 pt).}}

% Vimos que, para nosso esclarecimento, precisamos seguir os preceitos encontrados, já que Guimarães estabelece: "A valorização da palavra pela palavra encarna o objetivo precípuo do texto literário" (1985, p. 32) e, se isso não ficar bem esclarecido, nosso trabalho será seriamente prejudicado.

% Ou assim:

% Vimos que, para nosso esclarecimento, precisamos seguir os preceitos encontrados, já que ficou estabelecido que "a valorização da palavra pela palavra encarna o objetivo precípuo do texto literário" (GUIMARÃES, 1985, p.32) e, se isso não ficar bem esclarecido, nosso trabalho será seriamente prejudicado.

% As citações com mais de três linhas são chamadas de {\bf longas }e devem receber um destaque especial{\bf  }com recuo (reentrada) de {\bf 4cm }ou {\bf dezesseis toques}, da margem, mais {\bf cinco }toques para o início do parágrafo, além de usar fonte 10 pt, justificada.

% As citações longas, por já terem o destaque do recuo (reentrada), {\bf \underbar{não deverão ter aspas}} e o tamanho da fonte (letra) deve ser {\bf menor }que o do texto: {\bf {\it 10 pt.}}

% A distância entre as linhas do corpo da citação deve ser de um espaço {\bf simples}. Entre o texto da citação e o restante do trabalho, deve-se deixar {\bf dois {\it espaços duplos, }}antes e depois.

% Há uma certa dificuldade quanto ao reconhecimento de {\bf O}, {\bf A, OS, AS }como pronomes demonstrativos, mas essa dúvida é muito bem dirimida por Fernandes:

% \begin{citacao}
% Os pronomes O, A, OS e AS passam a ser pronomes demonstrativos sempre que numa frase puderem ser substituídos, sem alterar a estrutura dessa frase, respectivamente, por ISTO, ISSO, AQUILO, AQUELE, AQUELES, AQUELA, AQUELAS (1994, p. 19.).
% \end{citacao}


% Havendo {\bf supressão }de trechos {\bf dentro do texto }citado, faz-se a indicação com reticências entre colchetes {\bf [...]}: "Na comunicação diária, aquela comunicação que utilizamos no dia-a-dia, junto de nossos familiares e amigos, por exemplo, além da referencialidade da linguagem {\bf [...]} há pinceladas de função conativa" (CHALHUB , 1991, p. 37).

% No {\bf início }ou no {\bf fim }da citação, as reticências são usadas apenas quando o trecho citado {\bf não é uma sentença completa}. Entende-se por sentença completa aquela que o autor elaborou, com todos os seus elementos, isto é, uma sentença que contenha sujeito, predicado e seus complementos gramaticais exigidos. Caso contrário, {\bf se a sentença for completa, }no início ou no termino de citação, {\bf não se deve fazer }o uso das reticências. {\bf É {\it óbvio}} que se trata de parte de um todo, que se retirou um trecho, portanto, não há necessidade de se indicar com as reticências.

% Encerrava seu discurso nomeando os que figurariam somente nos exercício gerais, citando palavras de ordem, dentre as quais pudemos entender:

% ``... muitas mortes, desaparecimentos e desolação haverão de varrer este pais de norte a sul, de lesta a oeste e nada restará para a posteridade que sentirá a falta de um elo'' (MORGADO, 1967).

% Mais adiante, aquilo que mais chocou a todos quanto o ouviam:

% ``Arrasem com tudo, queimem tudo, ponham tudo abaixo, destruam com tudo, não poupem ninguém, nem crianças, nem mulheres, nem velhos...'' (MORGADO, 1967).

% Se a citação for usada para completar uma{\bf  }sentença do autor do Trabalho, esta terminará em vírgula e aquela iniciará {\bf sem a entrada de parágrafo} e com {\bf letra minúscula}. 

% A secretária ameaçou, dizendo que, ``da próxima vez, a máquina ficará sem as peças de reposição, se ele não chegar e disser o que precisa ser dito, uma vez que não estou aqui para servir de adivinha para seus caprichos desencontrados e sem nexo.'' (MARQUES, 1982, p. 34).

% Caso o texto do autor do Trabalho seja uma {\bf continuação }da citação, esta {\bf terminará por vírgula }e o texto reiniciado {\bf sem entrada de parágrafo e com letra minúscula.}

%  Os gramáticos são claros quando assumem uma posição quanto ao emprego do pronome oblíquo no início de oração. Cegalla (l 991, p. 419) diz claramente que:

% \begin{citacao}
%  Iniciar a frase com o pronome átono só é lícito na conversação familiar, despreocupada, ou na língua escrita, quando se deseja reproduzir a fala dos personagens, porém nós sabemos que na prática não é bem assim que acontece - as normas, rigorosamente, são esquecidas por quase todos os usuários do idioma falado, principalmente nas ocasiões informais.
% \end{citacao}

% Quando houver uma citação {\bf {\it dentro de outra citação, }}as aspas da segunda transformam-se em aspas simples ( ' ) (apóstrofo${}^{: }$Não confundir a palavra {\bf {\it apóstrofo}}{\it  que }é o sinal (`), com {\bf {\it apóstrofe}}{\it  que }é uma figura de linguagem que consiste na interpelação ou invocação do leitor, ouvinte ou outra pessoa no decorrer de um texto). Quando dentro da citação transcrita houver aspas, estas também são mudadas para aspas simples.



% Se for feita alguma {\bf interpelação, acréscimo }ou {\bf comentário }durante a citação, deve-se fazê-lo {\it entre colchetes }{\bf [ ]:}

% Também chamado de corpo do trabalho, [o desenvolvimento] tem por finalidade expor, demonstrar e fundamentar a explicitação do assunto a ser abordado. É normalmente dividido em seções ou capítulos, que variam de acordo com a natureza do assunto. (GARCIA, 2000, p. 17.).

% Se algum {\bf destaque }(grifo, negrito, itálico ou sublinhado) for dado, deve-se indicá-lo com a expressão {\bf grifo nosso, }entre colchetes:

% A primeira citação de uma obra deve ter sua referência bibliográfica completa. As subseqüentes citações da mesma obra {\bf podem ser referendadas de forma abreviada, }desde que não haja referências intercaladas de outras obras do mesmo autor (NBR 6023-2000) {\bf [grifo nosso].} 

% Caso o texto citado traga algum tipo de destaque dado pelo autor do trecho, devemos usar a expressão {\bf grifo do autor, entre colchetes.}

% A verdadeira felicidade é encontrada nos pequenos detalhes que vão se somando {\bf dia após dia }de convivência com o ser amado (GUERRERO, 2000, p. 12) {\bf [grifo do autor].}

% Quando o texto citado for composto por informações orais obtidas em aulas, palestras, debates, comunicações, etc. deve-se, entre parênteses, colocar a observação {\bf {\it informação oral, }}mencionando-se os dados disponíveis em nota de rodapé:

% Eichenberg constatou que, na costa do Rio Grande do Sul, especialmente no litoral norte, há a presença abundante de conformes fecais, especialmente nos meses do verão (informação oral). Essa presença tem causado graves transtornos a todos os veranistas.

% Se for o caso de se fazer menção a algo contido em {\it polígrafos, apostilas }ou quaisquer materiais avulsos, faz-se a indicação do nome do autor, quando for possível sua identificação, acrescentando-se a observação {\it `polígrafo', }`{\it material de propaganda', `panfleto', etc. }Procede-se da mesma forma com relação à data. Indica-se, se houver, caso contrário, registra-se s.d. (sem data). 



% \section{Citação de Citação}

% Se, num Trabalho, for feita uma citação de alguma passagem {\it já}{\bf {\it  }}{\it citada} em {\it outra obra, }a autoria deve ser referenciada pelo {\bf sobrenome do autor original }seguido da palavra latina {\bf apud }(que significa {\it segundo, conforme, de acordo com) }{\bf e o sobrenome do autor da obra consultada. }Dessa última, faz-se a referência completa (NBR6O23).

% ``O sistema consiste em colocar o recém-nascido no berço, ao lado da mãe, logo após o parto ou algumas horas depois, durante a estada de ambos na maternidade'' (HARUNARI apud GUARAGNA, 1992, p. 79).

% Temos aí palavras de Harunari que foram citadas por Guaranga e que estão sendo utilizadas, agora, no meu trabalho.

% {\bf Fonte:} FURASTÉ, Pedro Augusto. Normas Técnicas para o Trabalho Científico: explicitação das normas da ABNT. Porto Alegre: [s.n.], 2002. p. 49-56.

% \noindent 




% %%%%%%%%%%%%%%%%%%%%%%%%%%%%%%%%%%%%%%%%%%%%%%%%%%%%%%%%%%%%%%%%%%%%%%%%%%%%%%%%%%%
% % Referências 
% %%%%%%%%%%%%%%%%%%%%%%%%%%%%%%%%%%%%%%%%%%%%%%%%%%%%%%%%%%%%%%%%%%%%%%%%%%%%%%%%%%%
% %

% \bibliographystyle{abnt}

\bibliographystyle{abntex2-alf}


\bibliography{biblio} % arquivo que contém as referências (no formato bib). Colocar as suas lá (se tiver dúvida sobre como adicionar novas referências, usar o software JabRef ou Medley)



% \noindent {\\\bf Se tiver alguma dúvida, veja os exemplos seguintes:}\\

% \noindent {\bf \underbar{Monografia no todo}}\\

% \noindent {\bf Livros e Anais de Congresso (Autor. Título. Edição. Local de Publicação: editora, ano de publicação).}\\

% \noindent FURASTÉ, Pedro Augusto. {\bf Normas Técnicas para o Trabalho Científico}: explicitação das normas da ABNT. Porto Alegre: [s.n.], 2002. p. 49-56.

% \noindent BRADLEY, N. {\bf The XML Companion}. 3${}^{rd}$ ed. Boston: Addison-Wesley, 2002.

% \noindent FIELDS, D. K.; KDLB, M. A. {\bf Desenvolvendo na Web com JavaServer Pages}. Rio de Janeiro: Ciência Moderna, 2000.

% \noindent OLIVEIRA, R. S. de; CARISSIMI, A. da S.; TOSCANI, S. S. {\bf Sistemas Operacionais}. 2.ed. Porto Alegre: Instituto de Informática da UFRGS: Sagra Luzzatto, 2001. 247 p. (Série Livros Didáticos, n.11).

% \noindent SIMPÓSIO BRASILEIRO DE SISTEMAS MULTIMÍDIA E HIPERMÍDIA, SBMÍDIA, 7., 2001, Florianópolis. ... Florianópolis: UFSC: SBC, 2001.

% \noindent NATIONAL CONFERENCE ON ARTIFICIAL INTELLIGENCE, AAII, 17., 2000. {\bf Proceedings}... Menlo Park, CA: AAAI Press: The MIT Press, 2000.

% \noindent ~

% \noindent {\bf \underbar{Parte de Monografia}}\\

% \noindent {\bf Capítulo (Autor do capítulo. Título do capítulo. In: Autor/Editor/Organizador do livro. Título do livro. Edição. Local de publicação: editora, ano de publicação).}

% \noindent LUBASZEWSKI, M.; COTA, E. F.; KRUG, M. R. Teste e Projeto Visando o Teste de Circuitos e Sistemas Integrados. In: REIS, R. A. da L. (Ed.) {\bf Concepção de Circuitos Integrados}. 2.ed. Porto Alegre: Instituto de Informática da UFRGS: Sagra Luzzatto, 2002. p. 167-189.

% \noindent ROESLER, V.; BRUNO, G. G.; LIMA, J. V. de. ALM: Adaptative Layering Multicast. In: SIMPÓSIO BRASILEIRO DE SISTEMAS MULTIMÍDIA, SBMÍDIA, 7., 2001, Florianópolis. {\bf Anais...} Florianópolis: UFSC: SBC, 2001. p. 107-121.

% \noindent PFEFFER, A.; KOLLER, D. Semantics and Inference for Recursive Probability Models. In: NATIONAL CONFERENCE ON ARTIFICIAL INTELLIGENCE, AAII, 17., 2000. {\bf Proceedings... }Menlo Park, CA: AAAI Press: The MIT Press, 2000.

% \noindent ~

% \noindent {\bf \underbar{Dissertações, teses, trabalhos individuais, etc.}}\\

% \noindent MENEGHETTI, E. A. {\bf Uma Proposta de Uso da Arquitetura Trace como um Sistema de Detecção de Intrusão}. 2002. 105 f. Dissertação ( Mestrado em Ciência da Computação ) -- Instituto de Informática, UFRGS, Porto Alegre.

% \noindent SABADIN, R. da S. {\bf QoS em Serviços de Suporte por Frame Relay}. 2000. 35 f. Trabalho Individual ( Mestrado em Ciência da Computação ) -- Instituto de Informática, UFRGS, Porto Alegre.

% \noindent OTERO, I. M. {\bf Desenvolvimento de Sistema Cliente-Servidor em Camadas Utilizando Software Livre}. 2003. 55 f. Projeto de Diplomação ( Bacharelado em Ciência da Computação ) -- Instituto de Informática, UFRGS, Porto Alegre.

% \noindent ~

% \noindent {\bf \underbar{Artigo de periódico}}\\

% \noindent GONÇALVES, L. M. G.; CESAR JUNIOR, R. M. Robótica, Sistemas Sensorial e Motos: principais tendências e direções. {\bf Revista de Informática Teórica e Aplicada}, Porto Alegre, v.9, n.2, p. 7-36, out. 2002.

% \noindent JANOWIAK, R. M. Computers and Communications: a symbiotic relationship. {\bf Computer}, New York, v.36, n.1, p. 76-79, Jan. 2003.

% \noindent ~

% \noindent {\bf \underbar{Em meio eletrônico}}\\

% \noindent LISBOA FILHO, J.; IOCHPE, C.; BORGES, K. Reutilização de Esquemas de Bancos de Dados em Aplicações de Gestão Urbana. {\bf IP -- Informática Pública}, Belo Horizonte, v.4, n.1, p.105-119, June 2002. Disponível em: $<$http://www.ip.pbh.gov.br/ip0401.html $>$. Acesso em: set. 2002.

% \noindent ~

% \noindent {\bf \underbar{RFC}}\\

% \noindent CALLAGHAN, B.; PAWLOWSKI, B.; STAUBACH, P. {\bf NFS Version 3 Protocol Specification}: RFC 1831. [S.l.]: Internet Engineering Task Force, Network Working Group, 1995.

% \noindent ~

% \noindent {\bf \underbar{Norma}}\\

% \noindent INSTITUTE OF ELECTRICAL AND ELECTRONIC ENGINEERING. {\bf IEEE 1003.1c-1995}: information technology -- portable operating system interface (POSIX), threads extension [C language]. New York, 1995.

% \noindent ~

% \noindent {\bf \underbar{Observações}}\\

% Quando existirem mais de três autores, indica-se apenas o primeiro, acrescentando-se a expressão et al. Ex.: URANI, A. et al. Em casos em que a menção dos nomes for indispensável para certificar a autoria é facultado indicar todos os nomes.

% Em caso de autoria desconhecida, a entrada é feita pelo título. Ex.: DIAGNÓSTICO do Setor Editorial Brasileiro. São Paulo: Câmara Brasileira do Livro, 1993.

% Quando houver uma indicação de edição, esta deve ser transcrita, utilizando-se abreviaturas dos numerais ordinais e da palavra edição, ambas na forma adotada na língua do documento.

% Ex.: SCHAM, D. {\bf Schawm's Outline of Theory and Problems}. 5${}^{th}$ ed. New York: Schawm Publishing, 1956.

% PEDROSA, I. {\bf Da Cor a Cor Inexistente}. 6. ed. Rio de Janeiro: L. Cristiano, 1995.

% Não sendo possível determinar o local (cidade) de publicação, utiliza-se à expressão sine loco, abreviada, entre colchetes [S.l.].

% Quando a editora não puder ser indicada, deve-se indicar a expressão sine nomine, abreviada, entre colchetes [s.n].

% Quando o local e a editora não puderem ser identificados, utilizam-se [S.l.:s.n].




\end{document}
